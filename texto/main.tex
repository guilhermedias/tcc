\documentclass[
	% Opções da classe memoir
	12pt,				        % tamanho da fonte
	openany,			        % capítulos começam em pág ímpar (insere página vazia caso preciso)
	oneside,			        % para impressão em recto e verso. Oposto a oneside
	a4paper,			        % tamanho do papel. 
	% Opções da classe abntex2
	%chapter=TITLE,		        % títulos de capítulos convertidos em letras maiúsculas
	%section=TITLE,		        % títulos de seções convertidos em letras maiúsculas
	%subsection=TITLE,	        % títulos de subseções convertidos em letras maiúsculas
	%subsubsection=TITLE,       % títulos de subsubseções convertidos em letras maiúsculas
	% Opções do pacote babel
	english,			        % idioma adicional para hifenização
	brazil				        % o último idioma é o principal do documento
	]{abntex2}

% ---
% Pacotes básicos 
% ---
\usepackage{lmodern}			% Usa a fonte Latin Modern			
\usepackage[T1]{fontenc}		% Selecao de codigos de fonte.
\usepackage[utf8]{inputenc}		% Codificacao do documento (conversão automática dos acentos)
\usepackage{indentfirst}		% Indenta o primeiro parágrafo de cada seção.
\usepackage{color}				% Controle das cores
\usepackage{graphicx}			% Inclusão de gráficos
\usepackage{microtype} 			% para melhorias de justificação
% ---
		

% ---
% Pacotes de citações
% ---
% \usepackage[alf]{abntex2cite}						% Citações padrão ABNT
\usepackage[alf, abnt-nbr10520=1988]{abntex2cite}	% Citações em sentence case


% ---
% Informações para CAPA e FOLHA DE ROSTO
% ---
\titulo{Aprendizagem de máquina para previsão de desfechos em saúde mental: uma introdução para psicólogos}
\autor{Guilherme Augusto Dias}
\local{Porto Alegre}
\data{2024}
\orientador{Prof. Dr. Wagner de Lara Machado}
% \coorientador{}
% \instituicao{
%   Pontifícia Universidade Católica do Rio Grande do Sul
%   \par
%   Escola de Ciências da Saúde e da Vida
%   \par
%   Curso de Psicologia}
\tipotrabalho{TCC}
\preambulo{Trabalho de Conclusão de Curso apresentado como requisito para a obtenção do grau de Bacharel em Psicologia pela Escola de Ciências da Saúde e da Vida da Pontifícia Universidade Católica do Rio Grande do Sul.}
% ---

% ---
% Configurações de aparência do PDF final
% ---

% Alterando o aspecto da cor azul
\definecolor{blue}{RGB}{41,5,195}

% Informações do PDF
\makeatletter
\hypersetup{
		pdftitle={\@title}, 
		pdfauthor={\@author},
    	pdfsubject={\imprimirpreambulo},
	    pdfcreator={LaTeX with abnTeX2},
		pdfkeywords={abnt}{latex}{abntex}{abntex2}{trabalho acadêmico}, 
		colorlinks=true,		% true: colored links; false: boxed links
    	linkcolor=black,		% color of internal links
    	citecolor=black,		% color of links to bibliography
    	filecolor=black,		% color of file links
		urlcolor=black,
		bookmarksdepth=4
}
\makeatother
% --- 


% ---
% Posiciona figuras e tabelas no topo da página quando adicionadas sozinhas
% em um página em branco. Ver https://github.com/abntex/abntex2/issues/170
\makeatletter
\setlength{\@fptop}{5pt} % Set distance from top of page to first float
\makeatother
% ---

% ---
% Possibilita criação de Quadros e Lista de quadros.
% Ver https://github.com/abntex/abntex2/issues/176
%
\newcommand{\quadroname}{Quadro}
\newcommand{\listofquadrosname}{Lista de quadros}

\newfloat[chapter]{quadro}{loq}{\quadroname}
\newlistof{listofquadros}{loq}{\listofquadrosname}
\newlistentry{quadro}{loq}{0}

% Configurações para atender às regras da ABNT
\setfloatadjustment{quadro}{\centering}
\counterwithout{quadro}{chapter}
\renewcommand{\cftquadroname}{\quadroname\space} 
\renewcommand*{\cftquadroaftersnum}{\hfill--\hfill}

\setfloatlocations{quadro}{hbtp} % Ver https://github.com/abntex/abntex2/issues/176
% ---


% --- 
% Espaçamentos entre linhas e parágrafos 
% --- 

% O tamanho do parágrafo é dado por:
\setlength{\parindent}{1.3cm}

% Controle do espaçamento entre um parágrafo e outro:
\setlength{\parskip}{0.2cm}  % tente também \onelineskip

% ---
% Compila o índice
% ---
\makeindex
% ---


% ----
% Início do documento
% ----
\begin{document}

% Seleciona o idioma do documento (conforme pacotes do babel)
\selectlanguage{brazil}

% Retira espaço extra obsoleto entre as frases.
\frenchspacing 

% ----------------------------------------------------------
% ELEMENTOS PRÉ-TEXTUAIS
% ----------------------------------------------------------
\pretextual

% ---
% CAPA
% ---
\imprimircapa
% ---


% ---
% FOLHA DE ROSTO
% (o * indica que haverá a ficha bibliográfica)
% ---
\imprimirfolhaderosto
% ---


% ---
% Ficha bibliografica
% ---

% Isto é um exemplo de Ficha Catalográfica, ou ``Dados internacionais de
% catalogação-na-publicação''. Você pode utilizar este modelo como referência. 
% Porém, provavelmente a biblioteca da sua universidade lhe fornecerá um PDF
% com a ficha catalográfica definitiva após a defesa do trabalho. Quando estiver
% com o documento, salve-o como PDF no diretório do seu projeto e substitua todo
% o conteúdo de implementação deste arquivo pelo comando abaixo:
%
% \usepackage{pdfpages}		% necessário para comando \includepdf
% \begin{fichacatalografica}
%     \includepdf{fig_ficha_catalografica.pdf}
% \end{fichacatalografica}

% \begin{fichacatalografica}
% 	\sffamily
% 	\vspace*{\fill}                             % Posição vertical
% 	\begin{center}                              % Minipage Centralizado
% 	\fbox{\begin{minipage}[c][8cm]{13.5cm}		% Largura
% 	\small
% 	\imprimirautor
% 	%Sobrenome, Nome do autor
	
% 	\hspace{0.5cm} \imprimirtitulo  / \imprimirautor. --
% 	\imprimirlocal, \imprimirdata-
	
% 	\hspace{0.5cm} \thelastpage p. : il. (algumas color.) ; 30 cm.\\
	
% 	\hspace{0.5cm} \imprimirorientadorRotulo~\imprimirorientador\\
	
% 	\hspace{0.5cm}
% 	\parbox[t]{\textwidth}{\imprimirtipotrabalho~--~\imprimirinstituicao,
% 	\imprimirdata.}\\
	
% 	\hspace{0.5cm}
% 		1. Palavra-chave1.
% 		2. Palavra-chave2.
% 		2. Palavra-chave3.
% 		I. Orientador.
% 		II. Universidade xxx.
% 		III. Faculdade de xxx.
% 		IV. Título 			
% 	\end{minipage}}
% 	\end{center}
% \end{fichacatalografica}
% ---


% ---
% FOLHA DE APROVAÇÃO
% ---

% Isto é um exemplo de Folha de aprovação, elemento obrigatório da NBR
% 14724/2011 (seção 4.2.1.3). Você pode utilizar este modelo até a aprovação
% do trabalho. Após isso, substitua todo o conteúdo deste arquivo por uma
% imagem da página assinada pela banca com o comando abaixo:
%
% \begin{folhadeaprovacao}
% \includepdf{folhadeaprovacao_final.pdf}
% \end{folhadeaprovacao}
%
\begin{folhadeaprovacao}
  \begin{center}
    {\ABNTEXchapterfont\large\imprimirautor}

    \vspace*{1.5cm}

    \begin{center}
      \ABNTEXchapterfont\bfseries\Large\imprimirtitulo
    \end{center}

    \vspace*{1.5cm}
    
    \hspace{.45\textwidth}
    \begin{minipage}{.5\textwidth}
        \imprimirpreambulo
    \end{minipage}%
   \end{center}

   \vspace*{1.5cm}
        
   \begin{center}
   Aprovado em 17 de julho de 2024.

   \vspace*{1.5cm}

   Banca examinadora:
   \end{center}

%    \assinatura{\textbf{\imprimirorientador} \\ Orientador} 
   \assinatura{\textbf{Professor} \\ Convidado 1}
   \assinatura{\textbf{Professor} \\ Convidado 2}
   \vspace*{1.5cm}
\end{folhadeaprovacao}
% ---


% ---
% Agradecimentos
% ---
\begin{agradecimentos}
Valeu!
\end{agradecimentos}
% ---


% ---
% RESUMOS
% ---
% Resumo em português
\setlength{\absparsep}{18pt} % ajusta o espaçamento dos parágrafos do resumo
\begin{resumo}
Desenvolvimentos na área de aprendizagem de máquina representam uma série de oportunidades para diversas áreas de pesquisa e de atuação profissional.
O uso de modelos de aprendizagem de máquina na predição de desfecho para tratamentos em saúde mental tem sido investigado e apresenta resultados promissores.
A aplicação desse tipo de modelo possibilitaria, por exemplo, maior assertividade na personalização de tratamentos psicoterápicos a nível individual.
Apesar dos benefícios potenciais, o tema ainda é pouco discutido nos espaços de pesquisa, formação e atuação em psicologia clínica. Este trabalho tem
por objetivo introduzir o tema de aprendizagem de máquina a psicólogos. Busca-se apresentar, de maneira acessível, os conceitos e terminologia básicos
em aprendizagem de máquina, proporcionando um panorama geral da área. Para complementar a exposição conceitual, é construído um exemplo a partir de dados
de um estudo sobre eficácia de intervenções em psicologia positiva para depressão via internet. Por fim, discutem-se algumas implicações do uso desse tipo
de tecnologia para a pesquisa e para a prática em psicoterapia.

 \textbf{Palavras-chave}: aprendizagem de máquina, psicoterapia, psicologia positiva, depressão.
\end{resumo}

% Resumo em inglês
\begin{resumo}[Abstract]
 \begin{otherlanguage*}{english}
   Developments in area of machine learning represent a series of opportunities for different areas of research and professional activity. The use of machine
   learning models in predicting outcomes for mental health treatments has been investigated and shows promising results. The application of this type of model would
   enable, for example, greater assertiveness in the personalization of psychotherapeutic treatments at an individual level. Despite the potential benefits, the topic
   is still little discussed in research, training and practice in clinical psychology. This work aims to introduce the topic of machine learning to psychologists. The
   aim is to present, in an accessible way, the basic concepts and terminology in machine learning, providing a general overview of the area. To complement the conceptual
   exposition, an example is constructed based on data from a study on the effectiveness of positive psychology interventions for depression via the internet. Finally,
   some implications of using this type of technology for research and practice in psychotherapy are discussed.
   \vspace{\onelineskip}
 
   \noindent 
   \textbf{Keywords}: machine learning, psychotherapy, positive psychology, depression.
 \end{otherlanguage*}
\end{resumo}


% ---
% LISTA DE ILUSTRAÇÕES
% ---
\pdfbookmark[0]{\listfigurename}{lof}
\listoffigures*
\cleardoublepage
% ---


% ---
% LISTA DE QUADROS
% ---
% \pdfbookmark[0]{\listofquadrosname}{loq}
% \listofquadros*
% \cleardoublepage
% ---


% ---
% LISTA DE TABELAS
% ---
% \pdfbookmark[0]{\listtablename}{lot}
% \listoftables*
% \cleardoublepage
% ---


% ---
% LISTA DE ABREVIATURAS E SIGLAS
% ---
% \begin{siglas}
%   \item[ABNT] Associação Brasileira de Normas Técnicas
%   \item[abnTeX] ABsurdas Normas para TeX
% \end{siglas}
% ---


% ---
% LISTA DE SÍMBOLOS
% ---
% \begin{simbolos}
%   \item[$ \Gamma $] Letra grega Gama
%   \item[$ \Lambda $] Lambda
%   \item[$ \zeta $] Letra grega minúscula zeta
%   \item[$ \in $] Pertence
% \end{simbolos}
% ---


% ---
% Sumário
% ---
\pdfbookmark[0]{\contentsname}{toc}
\tableofcontents*
\cleardoublepage
% ---


% ----------------------------------------------------------
% ELEMENTOS TEXTUAIS
% ----------------------------------------------------------
\textual

\section{Conceitos Básicos de Aprendizagem de Máquina}

\section{Conceitos Básicos de Aprendizagem de Máquina}

\section{Conceitos Básicos de Aprendizagem de Máquina}

\input{02-desenvolvimento/01-machine-learning/main.tex}

\input{02-desenvolvimento/02-parte-2/main.tex}



\section{Conceitos Básicos de Aprendizagem de Máquina}

\input{02-desenvolvimento/01-machine-learning/main.tex}

\input{02-desenvolvimento/02-parte-2/main.tex}





\section{Conceitos Básicos de Aprendizagem de Máquina}

\section{Conceitos Básicos de Aprendizagem de Máquina}

\input{02-desenvolvimento/01-machine-learning/main.tex}

\input{02-desenvolvimento/02-parte-2/main.tex}



\section{Conceitos Básicos de Aprendizagem de Máquina}

\input{02-desenvolvimento/01-machine-learning/main.tex}

\input{02-desenvolvimento/02-parte-2/main.tex}







\section{Conceitos Básicos de Aprendizagem de Máquina}

\section{Conceitos Básicos de Aprendizagem de Máquina}

\section{Conceitos Básicos de Aprendizagem de Máquina}

\input{02-desenvolvimento/01-machine-learning/main.tex}

\input{02-desenvolvimento/02-parte-2/main.tex}



\section{Conceitos Básicos de Aprendizagem de Máquina}

\input{02-desenvolvimento/01-machine-learning/main.tex}

\input{02-desenvolvimento/02-parte-2/main.tex}





\section{Conceitos Básicos de Aprendizagem de Máquina}

\section{Conceitos Básicos de Aprendizagem de Máquina}

\input{02-desenvolvimento/01-machine-learning/main.tex}

\input{02-desenvolvimento/02-parte-2/main.tex}



\section{Conceitos Básicos de Aprendizagem de Máquina}

\input{02-desenvolvimento/01-machine-learning/main.tex}

\input{02-desenvolvimento/02-parte-2/main.tex}







\section{Conceitos Básicos de Aprendizagem de Máquina}

\section{Conceitos Básicos de Aprendizagem de Máquina}

\section{Conceitos Básicos de Aprendizagem de Máquina}

\input{02-desenvolvimento/01-machine-learning/main.tex}

\input{02-desenvolvimento/02-parte-2/main.tex}



\section{Conceitos Básicos de Aprendizagem de Máquina}

\input{02-desenvolvimento/01-machine-learning/main.tex}

\input{02-desenvolvimento/02-parte-2/main.tex}





\section{Conceitos Básicos de Aprendizagem de Máquina}

\section{Conceitos Básicos de Aprendizagem de Máquina}

\input{02-desenvolvimento/01-machine-learning/main.tex}

\input{02-desenvolvimento/02-parte-2/main.tex}



\section{Conceitos Básicos de Aprendizagem de Máquina}

\input{02-desenvolvimento/01-machine-learning/main.tex}

\input{02-desenvolvimento/02-parte-2/main.tex}







\section{Conceitos Básicos de Aprendizagem de Máquina}

\section{Conceitos Básicos de Aprendizagem de Máquina}

\section{Conceitos Básicos de Aprendizagem de Máquina}

\input{02-desenvolvimento/01-machine-learning/main.tex}

\input{02-desenvolvimento/02-parte-2/main.tex}



\section{Conceitos Básicos de Aprendizagem de Máquina}

\input{02-desenvolvimento/01-machine-learning/main.tex}

\input{02-desenvolvimento/02-parte-2/main.tex}





\section{Conceitos Básicos de Aprendizagem de Máquina}

\section{Conceitos Básicos de Aprendizagem de Máquina}

\input{02-desenvolvimento/01-machine-learning/main.tex}

\input{02-desenvolvimento/02-parte-2/main.tex}



\section{Conceitos Básicos de Aprendizagem de Máquina}

\input{02-desenvolvimento/01-machine-learning/main.tex}

\input{02-desenvolvimento/02-parte-2/main.tex}







% ----------------------------------------------------------
% ELEMENTOS PÓS-TEXTUAIS
% ----------------------------------------------------------
\postextual
% ----------------------------------------------------------


% ----------------------------------------------------------
% REFERÊNCIAS
% ----------------------------------------------------------
\bibliography{references.bib}


% ----------------------------------------------------------
% GLOSSÁRIO
% ----------------------------------------------------------
%
% Consulte o manual da classe abntex2 para orientações sobre o glossário.
%
%\glossary

% ----------------------------------------------------------
% APÊNDICES
% ----------------------------------------------------------

% ---
% Inicia os apêndices
% ---
% \begin{apendicesenv}

% % Imprime uma página indicando o início dos apêndices
% \partapendices

% % ----------------------------------------------------------
% \chapter{Quisque libero justo}
% % ----------------------------------------------------------

% % ----------------------------------------------------------
% \chapter{Nullam elementum urna}
% % ----------------------------------------------------------

% \end{apendicesenv}
% ---


% ----------------------------------------------------------
% ANEXOS
% ----------------------------------------------------------

% ---
% Inicia os anexos
% ---
% \begin{anexosenv}

% % Imprime uma página indicando o início dos anexos
% \partanexos

% % ---
% \chapter{Morbi ultrices rutrum lorem}
% % ---

% % ---
% \chapter{Cras non urna sed feugiat cum}
% % ---


% % ---
% \chapter{Fusce facilisis lacinia dui}
% % ---

% \end{anexosenv}

%---------------------------------------------------------------------
% INDICE REMISSIVO
%---------------------------------------------------------------------
% \phantompart
% \printindex
%---------------------------------------------------------------------
\end{document}

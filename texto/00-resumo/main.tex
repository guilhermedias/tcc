\documentclass{article}

\providecommand{\keywords}[1]{
    \vspace{1em}
    \noindent \textbf{Palavras-chave:} #1
}

\usepackage{titling}
\predate{}
\postdate{}

\title{Aprendizagem de máquina para previsão de desfechos em saúde mental: uma introdução para psicólogos}

\author{Guilherme Augusto Dias \vspace{1em} \and Prof. Dr. Wagner de Lara Machado \\ Orientador}

\date{}

\begin{document}

\maketitle

Desenvolvimentos na área de aprendizagem de máquina representam uma série de oportunidades para diversas áreas de pesquisa e de atuação profissional.
O uso de modelos de aprendizagem de máquina na predição de desfecho para tratamentos em saúde mental tem sido investigado e apresenta resultados promissores.
A aplicação desse tipo de modelo possibilitaria, por exemplo, maior assertividade na personalização de tratamentos psicoterápicos a nível individual.
Apesar dos benefícios potenciais, o tema ainda é pouco discutido nos espaços de pesquisa, formação e atuação em psicologia clínica. Este trabalho tem
por objetivo introduzir o tema de aprendizagem de máquina a psicólogos. Busca-se apresentar, de maneira acessível, os conceitos e terminologia básicos
em aprendizagem de máquina, proporcionando um panorama geral da área. Para complementar a exposição conceitual, é construído um exemplo a partir de dados
de um estudo sobre eficácia de intervenções em psicologia positiva via internet. Por fim, discutem-se algumas implicações do uso desse tipo de tecnologia
para a pesquisa e para a prática em psicoterapia.

\keywords{aprendizagem de máquina, psicoterapia, psicologia positiva, depressão}

\end{document}

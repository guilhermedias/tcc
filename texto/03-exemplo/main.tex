\section{Construção de um exemplo}

Introdução à construção do exemplo.

\subsection{O estudo original}

O estudo original buscava avaliar o impacto de intervenções de psicologia positiva conduzidas via internet sobre a percepção de felicidade e
sintomas depressivos \cite{Woodworth2017}, uma tentativa de replicar os resultados obtidos em um trabalho anterior conduzido por Seeligman e
colaboradores \cite{Seligman2005}.

\paragraph{Participantes}

Os participantes foram recrutados por meio de anúncios em veículos de comunicação australianos: páginas web, jornais e uma estação de rádio
local. Um total de 295 participantes completou a fase inicial de pré-teste. O grupo era composto majoritariamente por mulheres ($85,06\%$),
com idades entre 18 e 83 anos ($M=43,76; SD=12,43$); a maior parte das participantes possuia nível de educação superior ($74,88\%$) e classificou
a própria renda como média ou acima da média ($76\%$) \cite{Woodworth2017, Collins2023}.

\paragraph{Intervenção}

A.

\paragraph{Controles}

A.

\paragraph{Desfechos}

A.

\subsection{Plano de análise de dados}

A.

\subsection{Resultados e discussão}

A.

\subsection{O que é aprendizagem de máquina?}

Aprendizagem de máquina é a área da ciência da computação que tem como objetivo geral o desenvolvimento de programas
de computador capazes de aprender a realizar uma tarefa sem serem explicitamente programados \cite{Bi2019, Theobald2021}.
Neste contexto, aprendizagem refere-se a aplicação de procedimentos estatísticos e computacionais sobre um conjunto
de informações empíricas, buscando alcançar melhorias de desempenho em uma determinada tarefa \cite{Theobald2021}.

Aprender trata-se, portanto, de ajustar os parâmetros de um modelo estatístico e computacional aos dados observados
de modo a maximizar o desempenho na tarefa em questão \cite{Bi2019}. Programas de computador baseados em aprendizagem
de máquina são capazes de identificar padrões de interação complexos entre variáveis em conjuntos de dados com alta
dimensionalidade para realizar tarefas de classificação, regressão, agrupamento e outras \cite{Theobald2021}.

Considere, por exemplo, um estudo observacional hipotético que investiga a relação entre características de personalidade
e o nível de satisfação profissional entre psicólogos. O estudo baseia-se no modelo dos cinco grandes fatores da personalidade
\cite{Hutz2018} e usa o instrumento da Bateria Fatorial da Personalidade para coleta de dados, registrando as pontuações obtidas
nas escalas de neuroticismo, extroversão, socialização, realização e abertura \cite{Sancineto2015}. Além disso, os participantes
do estudo reportam o próprio nível de satisfação profissional em uma escala que contém os seguintes valores: baixo, médio e alto.
O conjunto de dados coletados é como apresentado na tabela \ref{table:example-data}.

\begin{table}[h!]
    \centering
    \begin{tabular}{llllll}
     neuroticismo & extroversão & socialização & realização & abertura & satisfação  \\
     \hline
     1 & 1 & 2 & 4 & 5 & alto  \\
     \hline
     1 & 1 & 2 & 4 & 5 & baixo \\
     \hline
     1 & 1 & 2 & 4 & 5 & médio \\
     \hline
     1 & 1 & 2 & 4 & 5 & médio
    \end{tabular}
    \caption{Exemplo de dados coletados no estudo hipotético.}
    \label{table:example-data}
\end{table}

É possível utilizar esse conjunto de dados para construir um modelo de apredizagem de máquina preditivo. Um algoritmo processa o conjunto de dados,
identificando os padrões de interação existentes entre as variáveis preditoras (características de personalidade) e o desfecho de interesse (nível
de satisfação profissional). O conhecimento adquirido durante o processamento dos dados é codificado nos parâmetros de um modelo de aprendizagem de
máquina. O modelo pode então ser utilizado para fazer predições sobre o nível de satisfação profissional de um indivíduo qualquer a partir de suas
características de personalidade.

\subsection{Os tipos de aprendizagem de máquina}
As técnicas de aprendizagem de máquina podem ser organizadas de diferentes maneiras, incluindo classificação pela estratégia adotada durante o processo
de aprendizagem e pelo objetivo geral de aprendizagem \cite{Theobald2021, Ng2001}.

\subsubsection{Aprendizagem supervisionada, não supervisionada e por reforço}
As categegorias mais comumente usadas na descrição de modelos de aprendizagem de máquina dizem respeito à estratégia de aprendizagem adotada. O
modelo pode ser construído segundo uma abordagem de aprendizagem supervisionada, aprendizagem não supervisionada ou aprendizagem por reforço
\cite{Theobald2021, Bi2019}.

A aprendizagem supervisionada assemelha-se ao processo de aprendizagem adotado por seres humanos, onde o aprendiz identifica padrões a partir de
um conjunto de exemplos preparado por um tutor. Durante a fase de aprendizagem, o modelo é exposto a um conjunto de dados que contém informações
sobre o desfecho de interesse para cada uma das observações \cite{Theobald2021, Bi2019}. O acesso às informações de desfecho providas por um agente
externo confere o caráter de supervisão a este processo. Técnicas de aprendizagem de máquina para regressão e classificação (support vector machines,
árvores de decisão, redes neurais) pertencem a esta categoria \cite{Bi2019}. Um exemplo para a aplicação deste tipo de técnica no contexto da psicologia
clínica é o uso de dados de ensaios clínicos, onde o desfecho para cada paciente é conhecido, para a construção de um modelo capaz de predizer o
resultado da intervenção para novos pacientes.

Na aprendizagem não supervisionada, o conjunto de dados analisado não contém informações sobre o desfecho para
as observações; perde-se assim a característica de supervisão. Espera-se que a aplicação identifique, de maneira
autônoma, os padrões de relacionamento existentes entre as variáveis do conjunto de dados. Técnicas de aprendizagem de
máquina populares para tarefas de agrupamento e redução de dimensionalidade (k-means clustering, PCA, TSNE) pertencem
a esta categoria.

Na aprendizagem por reforço, as aplicações adquirem conhecimento a respeito da tarefa ao longo do tempo por meio da obtenção
de feedback sobre seu desempenho.

\subsection{A construção de uma aplicação de machine learning}
\subsubsection{Análise descritiva}
\subsubsection{Pré-processamento}
Na etapa de pré-processamento, busca-se preparar o conjunto de dados de treinamento, colocando-no em um estado adequado à técnica
de aprendizagem de máquina que se pretende utilizar. Tratamentos comumente realizados na etapa de pré-processamento são: seleção de
características, transformações, imputações e balanceamento de classes.

A seleção de características consiste em eliminar do conjunto de dados as variáveis que tenham pouca contribuição para a aprendizagem da tarefa.
Em um conjunto de dados onde todas as observações são de pessoas brasileiras, a variável de nacionalidade não contribui para a explicação do desfecho
que se busca prever, portanto pode ser removida.

Transformações são aplicadas de acordo com os requisitos da técnica de aprendizagem de máquina em uso. Por exemplo, algumas técnicas de aprendizagem
de máquina são suscetíveis à influência de variáveis com escala muito superior às demais; nesses casos é comum transformação das variáveis para uma escala
padronizada (medida em desvios padrão a partir da média).

\subsubsection{Treinamento do modelo}
\subsubsection{Validação do modelo}

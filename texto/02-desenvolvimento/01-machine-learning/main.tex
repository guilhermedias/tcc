\subsection{O que é aprendizagem de máquina?}

Aprendizagem de máquina é a área da ciência da computação que tem como objetivo geral o desenvolvimento de programas
de computador capazes de aprender a realizar uma tarefa sem serem explicitamente programados \cite{Bi2019, Theobald2021}.
Neste contexto, aprendizagem refere-se a aplicação de procedimentos estatísticos e computacionais sobre um conjunto
de informações empíricas, buscando alcançar melhorias de desempenho em uma determinada tarefa \cite{Theobald2021}.

Aprender trata-se, portanto, de ajustar os parâmetros de um modelo estatístico e computacional aos dados observados
de modo a maximizar o desempenho na tarefa em questão. Esse processo de aprendizagem é comumente chamado de treinamento
do modelo \cite{Bi2019}. Programas de computador baseados em aprendizagem de máquina são capazes de identificar padrões
de interação complexos entre variáveis em conjuntos de dados com alta dimensionalidade para realizar tarefas de classificação,
regressão, agrupamento e outras \cite{Theobald2021}.

Considere, por exemplo, um estudo observacional hipotético que investiga a relação entre características de personalidade
e o nível de satisfação profissional entre psicólogos. O estudo baseia-se no modelo dos cinco grandes fatores da personalidade
\cite{Hutz2018} e usa o instrumento da Bateria Fatorial da Personalidade para coleta de dados, registrando as pontuações obtidas
nas escalas de neuroticismo, extroversão, socialização, realização e abertura \cite{Sancineto2015}. Além disso, os participantes
do estudo reportam o próprio nível de satisfação profissional em uma escala que contém os seguintes valores: baixo, médio e alto.

É possível utilizar esse conjunto de dados para construir um modelo de apredizagem de máquina preditivo. Um algoritmo processa o conjunto de dados,
identificando os padrões de interação existentes entre as variáveis preditoras (características de personalidade) e o desfecho de interesse (nível
de satisfação profissional). O conhecimento adquirido durante o processamento dos dados é codificado nos parâmetros de um modelo de aprendizagem de
máquina. O modelo pode então ser utilizado para fazer predições sobre o nível de satisfação profissional de um indivíduo qualquer a partir de suas
características de personalidade.

\subsection{Os tipos de aprendizagem de máquina}
As técnicas de aprendizagem de máquina podem ser organizadas de diferentes maneiras, incluindo classificação pela estratégia adotada durante o processo
de aprendizagem e pelo objetivo geral de aprendizagem \cite{Theobald2021, Ng2001}.

\subsubsection{Aprendizagem supervisionada, não supervisionada e por reforço}
As categegorias mais comumente usadas na descrição de modelos de aprendizagem de máquina dizem respeito à estratégia de aprendizagem adotada. O
modelo pode ser construído segundo uma abordagem de aprendizagem supervisionada, aprendizagem não supervisionada ou aprendizagem por reforço
\cite{Theobald2021, Bi2019}.

A aprendizagem supervisionada assemelha-se ao processo de aprendizagem adotado por seres humanos, onde o aprendiz identifica padrões a partir de
um conjunto de exemplos preparado por um tutor. Durante a fase de aprendizagem, o modelo é exposto a um conjunto de dados que contém informações
sobre o desfecho de interesse para cada uma das observações. O acesso às informações de desfecho providas por um agente externo confere o caráter
de supervisão a este processo. Técnicas de aprendizagem de máquina para regressão e classificação (support vector machines, árvores de decisão,
redes neurais) pertencem a esta categoria \cite{Theobald2021, Bi2019}. Um exemplo para a aplicação deste tipo de aprendizagem é usar de dados de
ensaios clínicos, onde o desfecho para cada paciente é conhecido, na construção de um modelo capaz de predizer o resultado da intervenção para
novos pacientes \cite{Collins2023}.

Na aprendizagem não supervisionada, o conjunto de dados analisado não contém qualquer informações sobre desfecho de interesse. Espera-se que o modelo
identifique os padrões de relacionamento existentes entre as variáveis do conjunto de dados e gere agrupamentos ou projeções de maneira autônoma.
Técnicas de aprendizagem de máquina para tarefas de agrupamento e redução de dimensionalidade (k-means clustering, PCA, TSNE) pertencem a esta categoria
\cite{Theobald2021, Bi2019}. Um exemplo para a aplicação deste tipo de aprendizagem é investigar os padrões de comorbidade em uma determinada população
clínica \cite{Sanchez2019}.

Na aprendizagem por reforço, o modelo aprende através de repetidos ciclos de tentativa e erro. A cada ciclo de aprendizagem, o modelo recebe feedback sobre seu
desempenho na tarefa, o feedback é incorporado à base de conhecimento construída pelo modelo em ciclos passados e, assim, melhora seu desempenho progressivamente
\cite{Theobald2021, Bi2019}. Um exemplo para a aplicação deste tipo de aprendizagem é auxiliar a tomada de decisões de tratamento em condições crônicas como a
esquizofrenia \cite{Shortreed2010}.

\subsubsection{Modelos discriminativos e generativos}

Estratégias de aprendizagem supervisionada e não supervisionada podem ser utilizada na construção de modelos com objetivos de aprendizagem distintos. Modelos
discriminativos tem por objetivo modelar probabilidade condicional de um desfecho dadas determinadas condições \cite{Bi2019, Ng2001}. Um modelo discriminativo,
poderia representar diretamente a probabilidade de resposta a uma intervenção psicoterápica dadas as condicões socioeconômicas do paciente, como escolaridade e
renda. Modelos discriminativos são comumente usados em tarefas de regressão e classificação \cite{Bi2019}.

Modelos generativos buscam modelar a distribuição de probabilidade conjunta para as variáveis presentes no conjunto de dados, ou seja, a probabilidade associada
a cada combinação de variáveis observada no conjunto de dados de treinamento \cite{Bi2019, Ng2001}. Um modelo generativo poderia, por exemplo, representar a
probabilidade associada a cada combinação de escolaridade, renda e resposta à intervenção observada durante seu treinamento. A distribuição de probabilidade
conjunta completa representa, em certa medida, o processo subjacente de geração dos dados, o que permite que modelos generativos sejam utilizados para gerar
observações sintéticas \cite{Bi2019}. Esse tipo de modelo é associado a ferramentas de inteligência artificial generativa como o Chat GTP \cite{Wu2023}.

\subsection{A construção de uma aplicação de machine learning}

O processo para construção de modelos de aprendizagem de máquina pode variar de acordo com a abordagem adotada, mas, de modo geral, consiste na sequência de etapas
de pré-processamento, separação dos dados, treinamento e avaliação de desempenho do modelo \cite{Greener2021}.

\subsubsection{Pré-processamento}

O desempenho de um modelo de aprendizagem de máquina depende, em grande medida, da forma como o conjunto de dados é apresentado. Assim é fundamental uma etapa
de processamento inicial para garantir que os padrões mínimos de qualidade de dados são atendidos. Tarefas de pré-processemento comuns são imputação de dados
faltantes, balanceamento de classes através de \textit{up-sampling} ou \textit{down-sampling}, recodificação de variáveis categóricas e padronização
de variáveis quantitativas \cite{Delgadillo2020}.

\subsubsection{Separação dos dados}

Deve-se avaliar o desempenho do modelo resultante ao final do processo de aprendizagem. Uma avaliação efetiva deve verificar o comportamento do modelo quando
exposto a um conjunto de dados inéditos, permitindo uma boa estimativa de seu desempenho em um contexto naturalístico. Assim, uma parte dos dados disponíveis,
cerca de 10\%, deve ser reservada para a avaliação de desempenho. Esses dados são chamados conjunto de dados de teste e não devem ser usados em nenhuma das
etapas de treinamento \cite{Greener2021}. 

O processo de treinamento do modelo deve ser monitorado para evitar falhas de aprendizagem. Por exemplo, generalizações indevidas feita a partir de uma única
observação.  O monitoramento é feito a partir de avaliações de desempenho intermediárias, que acontecem durante o treinamento. É necessário, portanto reservar
uma parte dos dados restantes, cerca de 10\%, para fazer o monitoramento. Esses dados são chamados de dados de validação \cite{Greener2021}.

Uma alternativa para aumentar quantidade de dados disponíveis para treinamento é o uso da técnica de validação cruzada, sendo a \textit{k-fold cross-validation}
sua apresentação mais comum. Nesta abordagem, o conjunto de dados de treinamento é dividido em k partes de tamanhos iguais e são realizadas k rodadas de treinamento;
a cada rodada uma das partes é separada e usada como conjunto de dados de validação. Assim, evita-se a reserva de dados exclusivos para validação \cite{Delgadillo2020}.

\subsubsection{Treinamento}

O treinamento consiste na implementação e execução do algoritmo de aprendizagem de máquina responsável por treinar o modelo. A implementação geralmente faz uso de
uma linguagem de programação de alto nível e um framework de aprendizagem de máquina \cite{Greener2021}. Um exemplo de framework é o scikit-learn \cite{ScikitLearn}
para a linguagem de programação Python \cite{Python}.

O processo de treinamento envolve também a seleção de hiperparâmetros: parâmetros de configuração do algoritmo de treinamento. Hiperparâmetros recebem este nome para
diferenciá-los dos parâmetros do modelo em construção. Os hiperparâmetros selecionados governam o processo de aprendizagem e podem ter um impacto importante no desempenho
do modelo. A seleção de hiperparâmetros pode ser manual ou automatizada durante o processo de treinamento por meio da técnica de \textit{k-fold cross-validation}, onde,
em cada uma das rodadas de treinamento, uma configuração de hiperparâmetros é utilizada e avaliada \cite{Delgadillo2020}.

\subsubsection{Avaliação do desempenho}

A avaliação do desempenho de um modelo varia de acordo com a estratégia de aprendizagem. Para modelos de aprendizagem não supervisionada, é comum a utilização de medidas de
dispersão, homogeinedade e variância preservada \cite{Naeem2023}. Modelos de aprendizagem supervisionada que realizam tarefas de regressão são avaliados com medidas de erro
como o erro quadrático médio \cite{Delgadillo2020}. Modelos de aprendizagem supervisionada para tarefas de classificação utilizam métricas de acurácia preditiva como a área
sob a curva ROC \cite{Delgadillo2020}.
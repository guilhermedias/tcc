\subsection{O que é aprendizagem de máquina?}

Aprendizagem de máquina é a área da ciência da computação que tem como objetivo geral desenvolvimento
de programas de computador capazes de aprender sem serem explicitamente programados \cite{Bi2019, Theobald2021}.
Neste contexto, aprendizagem refere-se a aplicação de procedimentos estatísticos e computacionais sobre um conjunto
de informações empíricas, buscando alcançar melhorias de desempenho em uma determinada tarefa \cite{Theobald2021}.
Aprender trata-se, portanto, de ajustar os parâmetros de um modelo estatístico e computacional aos dados observados
de modo a maximizar o desempenho na tarefa em questão \cite{Bi2019}.
Programas de computador baseados em aprendizagem de máquina são capazes de identificar padrões de interação complexos
entre variáveis em conjuntos de dados com alta dimensionalidade para realizar tarefas de classificação, regressão, agrupamento
e outras \cite{Theobald2021}.

Considere, por exemplo, o cenário onde tem-se um conjunto de dados provenientes de um estudo clínico; os dados
incluem, para cada paciente, informações demográficas (idade, gênero, renda, educação), informações clínicas
coletadas antes do tratamento (respostas observadas em entrevistas ou testes psicológicos) e informações clínicas
coletadas após o tratamento, incluindo a avaliação final sobre a eficácia do tratamento. Uma aplicação de aprendizagem de
máquina seria capaz de ingerir tal conjunto de dados e aprender a realizar determinadas tarefas. Uma tarefa possível
é prever o desfecho da intervenção testada mediante a apresentação das informações demográficas e clínicas de um
novo paciente. Outra possibilidade é utilizar padrões de similaridade no cojunto de dados para criar agrupamentos,
facilitando a identificação de subgrupos dentro da população alvo e permitindo analisar os efeitos da intervenção
de maneira mais assertiva para os diferentes subgrupos.

\subsection{Quais os tipos de técnicas de machine learning?}
As diferentes técnicas de aprendizagem de máquina podem ser classificadas de acordo com a estratégia adotada
durante o processo de aprendizagem. As principais categegorias são: aprendizagem supervisionada, aprendizagem não
supervisionada e aprendizagem por reforço \cite{Bi2019}.

\subsubsection{Supervisionada}
Na aprendizagem supervisionada, a aplicação é exposta a um conjunto de dados que contém informações sobre o desfecho
para cada uma das observações. A informação sobre o desfecho pode ser um conjunto de variáveis ou uma única variável;
em ambos os casos as variáveis podem representar quantidades ou categorias. O fato de a aplicação tem acesso às informações
de desfecho, geralmente providas por um agente externo (um pesquisador que registrou os desfechos manualmente), confere o
caráter de supervisão à este processo. Técnicas de aprendizagem de máquina para regressão e classificação (support vector
machines, árvores de decisão, redes neurais) pertencem a esta categoria.

\subsubsection{Não supervisionada}
Na aprendizagem não supervisionada, o conjunto de dados analisado não contém informações sobre o desfecho para
as observações; perde-se assim a característica de supervisão. Espera-se que a aplicação identifique, de maneira
autônoma, os padrões de relacionamento existentes entre as variáveis do conjunto de dados. Técnicas de aprendizagem de
máquina populares para tarefas de agrupamento e redução de dimensionalidade (k-means clustering, PCA, TSNE) pertencem
a esta categoria.

\subsubsection{Por reforço}
Na aprendizagem por reforço, as aplicações adquirem conhecimento a respeito da tarefa ao longo do tempo por meio da obtenção
de feedback sobre seu desempenho.

\subsection{A construção de uma aplicação de machine learning}
\subsubsection{Análise descritiva}
\subsubsection{Pré-processamento}
Na etapa de pré-processamento, busca-se preparar o conjunto de dados de treinamento, colocando-no em um estado adequado à técnica
de aprendizagem de máquina que se pretende utilizar. Tratamentos comumente realizados na etapa de pré-processamento são: seleção de
características, transformações, imputações e balanceamento de classes.

A seleção de características consiste em eliminar do conjunto de dados as variáveis que tenham pouca contribuição para a aprendizagem da tarefa.
Em um conjunto de dados onde todas as observações são de pessoas brasileiras, a variável de nacionalidade não contribui para a explicação do desfecho
que se busca prever, portanto pode ser removida.

Transformações são aplicadas de acordo com os requisitos da técnica de aprendizagem de máquina em uso. Por exemplo, algumas técnicas de aprendizagem
de máquina são suscetíveis à influência de variáveis com escala muito superior às demais; nesses casos é comum transformação das variáveis para uma escala
padronizada (medida em desvios padrão a partir da média).

\subsubsection{Treinamento do modelo}
\subsubsection{Validação do modelo}

\subsection{O que é machine learning?}

Aprendizagem de máquina é a área da ciência da computação que tem por objetivo o desenvolvimento
de aplicações capazes de aprender estratégias para execução de uma determinada tarefa a partir da
exposição a informações relevantes, diferenciando-se de sistemas tradicionais onde o comportamento
deve ser pré-programado \cite{Bi2019}. Aplicações baseadas em aprendizagem de máquina são capazes de
identificar padrões de interação complexos entre variáveis de um conjunto de dados com alta dimensionalidade
para realizar tarefas de classificação, regressão, agrupamento e outras \cite{Delgadillo2020}.

Por trás dessas aplicações, encontram-se modelos computacionais com múltiplos parâmetros e o processo de
aprendizagem consiste em encontrar um conjunto de parâmetros que produz os melhores resultados para a tarefa
em questão; nesse sentido, a aprendizagem de máquina aproxima-se de métodos estatísticos tradicionais. A diferenciação
entre métodos estatísticos e de aprendizagem de máquina é feita, muitas vezes, com base em critérios históricos,
mas é possível identificar características próprias das abordagens de aprendizagem de máquina: ao contrário da abordagem
estatística inferencial, por exemplo, a aprendizagem de máquina não assume nenhuma relação entre variáveis; em certa medida,
pode-se dizer que é uma abordagem agnóstica de teoria \cite{Bi2019, Delgadillo2020}.

\subsection{Quais os tipos de técnicas de machine learning?}
As diferentes técnicas de aprendizagem de máquina podem ser classificadas de acordo com a estratégia adotada
durante o processo de aprendizagem. As principais categegorias são: aprendizagem supervisionada, aprendizagem não
supervisionada e aprendizagem por reforço.

\subsubsection{Supervisionada}
Na aprendizagem supervisionada, a aplicação é exposta a um conjunto de dados que contém informações sobre o desfecho
para cada uma das observações. Técnicas de aprendizagem de máquina para classificação e regressão (support vector machines,
árvores de decisão, redes neurais) pertencem a esta categoria.

\subsubsection{Não supervisionada}
Na aprendizagem não supervisionada, o conjunto de dados analisado não contém informações sobre o desfecho para
as observações. A aplicação é responsável por identificar, de maneira autônoma, os padrões de relação e similaridades existentes
no conjunto de dados. Técnicas de aprendizagem de máquina populares para tarefas de agrupamento e redução de dimensionalidade
(k-means clustering, PCA, TSNE) pertencem a esta categoria.

\subsubsection{Por reforço}
Na aprendizagem por reforço, as aplicações adquirem conhecimento a respeito da tarefa ao longo do tempo por meio da obtenção
de feedback sobre seu desempenho.

\subsection{A construção de uma aplicação de machine learning}
\subsubsection{Análise descritiva}
\subsubsection{Pré-processamento}
Na etapa de pré-processamento, busca-se preparar o conjunto de dados de treinamento, colocando-no em um estado adequado à técnica
de aprendizagem de máquina que se pretende utilizar. Tratamentos comumente realizados na etapa de pré-processamento são: seleção de
características, transformações, imputações e balanceamento de classes.

A seleção de características consiste em eliminar do conjunto de dados as variáveis que tenham pouca contribuição para a aprendizagem da tarefa.
Em um conjunto de dados onde todas as observações são de pessoas brasileiras, a variável de nacionalidade não contribui para a explicação do desfecho
que se busca prever, portanto pode ser removida.

Transformações são aplicadas de acordo com os requisitos da técnica de aprendizagem de máquina em uso. Por exemplo, algumas técnicas de aprendizagem
de máquina são suscetíveis à influência de variáveis com escala muito superior às demais; nesses casos é comum transformação das variáveis para uma escala
padronizada (medida em desvios padrão a partir da média).

\subsubsection{Treinamento do modelo}
\subsubsection{Validação do modelo}

\subsection{O que é machine learning?}

Aprendizagem de máquina é a área da ciência da computação que tem por objetivo o desenvolvimento
de aplicações capazes de aprender estratégias para execução de uma determinada tarefa a partir de
experiências, diferenciando-se de sistemas tradicionais onde o comportamento deve ser pré-programado.

Por trás dessas aplicações, encontram-se modelos computacionais com múltiplos parâmetros e o processo de
aprendizgem consiste em encontrar um conjunto de parâmetros que produz os melhores resultados para a tarefa
em questão. Nesse sentido, a aprendizagem de máquina, aproxima-se de métodos estatísticos tradicionais como
regressões.

\subsection{Quais os tipos de machine learning?}
As diferentes técnicas de aprendizagem de máquina podem ser classificadas de acordo com a estratégia adotada
para o processo de aprendizagem. As principais categegorias são: aprendizagem supervisionada, aprendizagem não
supervisionada e aprendizagem por reforço.

\subsubsection{Supervisionada}
Na aprendizagem supervisionada, o processo de aprendizagem consiste em expor o modelo a um conjunto de observações
rotuladas.

\subsubsection{Não supervisionada}
Na aprendizagem supervisionada, o processo de aprendizagem consiste em expor o modelo a um conjunto de observações
não rotuladas.

\subsubsection{Por reforço}

\subsection{O processo de construção de um modelo de machine learning}
\subsubsection{Análise exploratória}
\subsubsection{Tratamento dos dados}
\subsubsection{Separação do conjunto de dados}
\subsubsection{Treinamento do modelo}
\subsubsection{Validação do modelo}

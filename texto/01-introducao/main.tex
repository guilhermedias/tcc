\section{Introdução}

Diversos modelos de psicoterapia apresentam evidência de eficácia no tratamento de transtornos mentais. Uma parcela significativa dos pacientes,
no entanto, não responde às intervenções, podendo até mesmo apresentar piora quando submetida a tratamento psicoterápico \cite{Cuijpers2021}. A
variabilidade na taxa de resposta ao tratamento é parcialmente explicada pelo quadro clínico do paciente; ainda assim, observou-se que pessoas
de uma mesma população clínica respondem de maneiras diferentes a um mesmo tratamento \cite{Hofmann2012}. Especula-se que um atendimento personalizado
a nível individual possa melhorar a taxa de resposta ao tratamento \cite{Norcross2010, Norcross2018}.

A literatura aponta uma série de variáveis preditoras de desfecho clínico para pacientes em tratamento psicoterápico, o que permitiria a adaptação
dos protocolos de intervenção de acordo com a expectativa de resposta \cite{Smagula2019, Andover2020}. Não existe, porém, consenso acerca do valor
preditivo de cada variável a nível individual e interações complexas entre diferentes preditores podem prejudicar a acurácia das previsões, podendo
impactar propostas de personalização de tratamento negativamente \cite{Taubitz2022}.

Técnicas de inteligência artificial baseadas em aprendizagem de máquina apresentam a capacidade de integrar uma grande quantidade de dados, impondo
poucas restrições ao comportamento das variáveis observadas e produzindo modelos flexíveis aplicáveis em diferentes contextos \cite{Dwyer2018}. Em essência,
a aprendizagem de máquina consiste no uso de métodos estatísticos e computacionais para identificar padrões de relacionamento subjacentes a um grande conjunto
de dados, permitindo a construção de modelos classificatórios ou preditivos \cite{Roth2018}. Estudos sobre a aplicação de modelos de aprendizagem de máquina na
previsão do desfecho de tratamentos em saúde mental apresentam resultados promissores \cite{Dwyer2018}, o que possibilitaria maior assertividade na personalização
de tratamentos psicoterápicos a nível individual. Embora promissores, os modelos de aprendizagem de máquina apresentam uma série de limitações e sua aplicação
em contextos de saúde deve ser criteriosamente avaliada. Para tanto, a Organização Mundial da Saúde documentou uma série de considerações sobre o uso dessa tecnologia
nos cuidados em saúde, incluindo temas como validação analítica do modelo, métricas de performance e comparação \cite{WHO2023}.

Este trabalho tem por objetivo identificar e discutir as possíveis aplicações e limitações do uso de modelos baseados em aprendizagem de máquina na prática da psicologia
clínica. Busca-se também ilustrar o uso da aprendizagem de máquina no auxílio à tomada de decisões para o planejamento de intervenções psicoterápicas. Utilizando um
conjunto de dados de uma intervenção digital em psicologia positiva para depressão \cite{Collins2023}, pretende-se construir modelos de regressão linear, clustering, support
vector machine, árvores de decisão e redes neurais, avaliando o desempenho e discutindo a adequação de cada técnica ao contexto da psicologia clínica.
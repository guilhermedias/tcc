\chapter{Conclusão}

Este trabalho teve como objetivo principal introduzir o tema de aprendizagem de máquina para psicólogos. Foi exposto
um panorama geral com a apresentação dos conceitos fundamentais da área. Buscando complementar a exposição conceitual,
construiu-se um exemplo concreto a partir de dados de um estudo de eficácia de intervenção em psicologia positiva para
depressão.

Observou-se que o desenvolvimento de ferramentas de aprendizagem de máquina cada vez mais acessíveis e a ampla disseminação
do conhecimento relacionado à área contribuem positivamente para o avanço desta tecnologia no campo da psicologia. O uso de
tecnologias baseadas em aprendizagem de máquina no auxílio a tomada de decisões clínicias em psicoterapia mostra-se uma
possibilidade tecnicamente viável com potencial de melhorar a qualidade e eficiência dos serviços oferecidos.

Uma das principais limitações do estudo é o acesso a um conjunto de dados de treinamento com volume e qualidade adequados.
Técnicas avançadas de pré-processamento e treinamento poderiam, em parte, mitigar o impacto desta limitação, mas seu uso foi
descartado com o objetivo de manter a construção simples e didática. Assim, a escassez de dados para treinamento do modelo
limitou o desempenho do modelo construído durante a construção do exemplo.

Trabalhos futuros incluem a criação de diretrizes para coleta e avaliação da qualidade de dados para desenvolvimento de modelos de
aprendizagem de máquina na psicologia, atuando diretamente sobre um dos maiores desafios da pesquisa na área, a obtenção de dados
adequados. Deve-se também investir no desenvolvimento de mecanismos de inspeção para modelos complexos de alto desempenho como as
redes neurais artificiais, uma condição necessária para a utilização da tecnologia na prática clínica.

O avanço no desenvolvimento e disseminação das tecnologias de aprendizagem de máquina representa uma oportunidade de melhoria
na pesquisa e atuação profissional em psicoterapia. Deve-se investir na adaptação das técnicas ao contexto de tratamentos em saúde
mental para realizar o potencial benéfico que a aplicação da tecnologia de aprendizagem de máquina oferece.